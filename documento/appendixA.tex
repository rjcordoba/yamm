\def\separadorins{\vskip5pt\centerline{\vbox{\hrule height .05pt width 4.1in}}\vskip33pt}

%% \def\separadorins{\vskip55pt}

\noindent Un programa \c-- consiste en un conjunto de posiciones, cada una de las cuales
contiene una instrucción o está vacía. Una posición, o bien es la primera, o bien es la siguiente de
otra posición. Llamamos interpretar una posición ψ a ejecutar la instrucción que hay en ψ o parar
la ejecución del programa si ψ está vacía. La ejecución de un programa empieza interpretando la
primera posición del mismo.

A continuación se muestran las instrucciones que pueden aparecer en una posición. Se usa una
variante de la forma de Backus-Naur para definir la sintaxis, con los no-terminales entre los signos
{\fcode«⟨»} y {\fcode«⟩»}, los terminales en negrita, la cabecera de la producción separada del
cuerpo por el signo «→» y las opciones de sustitución separadas por el signo «|». Seguido de la
definición sintáctica de las instrucciones se da la explicación para la ejecución de cada
instrucción.
\vskip30pt
\moveright9pt\vbox{\instenc \hbox{\hskip.3em Instrucciones\hskip.3em}\kern.3ex\hrule}
\vskip32pt
\vbox{
\insc--{marcar}1%
Se pone el símbolo «1» en la posición referida por la instrucción si no hay nada en dicha posición,
o un símbolo «’» si la posición no está vacía. La posición referida por una instrucción de este tipo
es:
\listainsc1{marcar}%
Si la instrucción está en la posición ψ, después de ejecutar esta instrucción se pasa a interpretar
la posición siguiente a ψ.
}
\separadorins
\vbox{
\insc--{borrar}0%
Se quitan todos los símbolos «’» de la posición referida, y el símbolo «1» si lo hubiera. La
posición referida por una instrucción de este tipo es:
\listainsc0{borrar}%
Si la instrucción está en la posición ψ, después de ejecutar esta instrucción se pasa a interpretar
la posición siguiente a ψ.
}
\vfil\eject
\vbox{
\insc--{comparar}{\raise.1ex\hbox{=}}%
Sea ψ la posición en la que está la instrucción. Si la posición referida por la instrucción y la
primera posición no son iguales según la definición de igualdad de posiciones que se da a
continuación, se interpreta la posición siguiente a ψ; si son iguales se interpreta la posición
siguiente a la siguiente posición de ψ.  La posición referida por una instrucción de este tipo es:
\listainsc={comparar}%
Dos posiciones son iguales si y sólo si una de las siguientes afirmaciones es cierta:
\listanormal
En ninguna de las dos posiciones hay nada.\par
En cada una de las dos posiciones hay solamente un símbolo.\par
Cada una de las dos posiciones tiene símbolos «’» y si se quita a ambas uno de estos símbolos «’» las posiciones son iguales.
\finlista
}
\separadorins
\vbox{
\insc--{saltar}{\font\festrella="DejaVuSansMono" at 26.76pt\raise.1ex\hbox{\estrella}}%
Se interpreta la posición referida por la instrucción. La posición referida por una instrucción de
este tipo es:
\listainsc{\noindent\estrella}{saltar}
}
