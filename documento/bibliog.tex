\font\fnombre="TeXGyreSchola-Regular:+smcp" at 13pt
\font\fff="TeXGyreSchola-Regular:slant=0.2" at 13pt
\font\urtea="TeXGyreSchola-Regular:mapping=tex-text" at 13pt
\def\libro#1(#2:#3.{\advance\count1 by 1 {\fnombre#1}({\urtea#2}:{\fff#3\/}.}
\hbox{}\vskip.2in
\centerline{{\ftextoappendix Bibliografía}}
\vskip.5in
\def\pie{Bibliografía}
\begingroup
\everypar={\llap{[\number\count1]\hskip10pt}}\leftskip20pt \parskip14pt \rm \count1=0 \parindent=0pt \baselineskip=20pt

\libro Alan M. Turing (1936): On computable numbers, with an application to the
entscheidungsproblem. Incluído en «The essential Turing», Oxford University Press, 2004, editado por
B.~Jack Copeland.

\libro Stephen C. Kleene (1952): Introduction to Metamathematics. Ishi Press,\kern-.4em\break 2009.

\libro A.J. Kfoury, Robert N. Moll {\urtea y} Michael A. Arbib (1982): A programming approach to
computability. Springer-Verlag.

\libro Georg Cantor (1895--1897): Contributions to the founding of the theory of transfinite
numbers. Traducido y preparado por Philip E. B. Jourdain. Dover, 1955.

\libro David Hilbert {\urtea y} Wilhelm Ackermann (1938): Principles of mathema\-tical logic. Chelsea
Publishing, 1950.

\libro George S. Boolos, John P. Burgess {\urtea y} Richard C. Jeffrey (2007): Computability and
logic. Cambridge University Press.

\libro Alfred V. Aho, Monica S. Lam, Ravi Sethi {\urtea y} Jeffrey D. Ullman (2006): Compilers:
Principles, Techniques, and Tools. 	Pearson Education, Inc.

\libro Brian W. Kernighan {\urtea y} Dennis M. Ritchie (1988): The C Programming Language. Prentice Hall.

\endgroup

