\encabezadop{Sobre el trabajo}
El presente trabajo trata sobre la equivalencia de distintos tipos de modelos de cómputo; más
concretamente se centra en demostrar que la potencia de cómputo de los lenguajes actuales, las
máquinas de Turing y las funciones recursivas es igual, esto es, que cualquier problema que se puede
resolver en cualquiera de estos modelos se puede resolver también en los otros. En toda la
explicación se da por hecho que se conoce la teoría sobre las máquinas de Turing tal y como las
definió originalmente su creador en [1]\footnote*{{\fpies \vbox to 14pt{}Las indicaciones [{\it x}] hacen referencia a
libros citados en la bibliografía.}} y las reglas para definir funciones recursivas según están
presentadas por S.~C. Kleene en [5].

\def\footnoterule{\kern-2pt \hrule}
\def\lambda{λ-\hskip0.1em}
\encabezados{Justificación y explicación}
Las máquinas de Turing y el \lambda cálculo inventado por Alonzo Church, así como la teoría sobre las
funciones recursivas, nacieron en el siglo XX en el marco de los estudios de lógica y el límite
de lo resoluble. La tesis Church-Turing nos dice que cualquier problema para el que exista un
procedimiento que lo solucione puede resolverse tanto con una máquina de Turing como usando \lambda
cálculo.

La tesis no se puede demostrar pero tampoco ha sido refutada, con lo que sabemos que no hay método
conocido más potente que estos dos para expresar soluciones a problemas. Cuando un lenguaje tiene la
misma capacidad que ellos para definir algoritmos se dice que es Turing-\kern0.1em completo, esto
es, que puede definirse con él una solución para cualquier problema que pueda resolver una máquina
de Turing. Sin embargo a la mayoria de personas que empieza a estudiar computación la duda sobre la
capacidad de cómputo le surge en sentido contrario: ¿cómo es posible con un método tan simple como
una máquina de Turing resolver lo que resuelve un ordenador actual?

Cuando leemos un tratado sobre una pintura conocida, o una demostración del teorema de Pitágoras
posiblemente sea por lo interesante que resulta admirar desde otros puntos de vista las obras e
ideas que nos gustan, y en todo caso nos da una excusa para reflexionar sobre ellas. El hecho de que
existan tantas obras sobre un mismo tema y docenas de demostraciones distintas del teorema de
Pitágoras es buena evidencia. Este trabajo no contiene algo novedoso ni especialmente interesante;
es un acercamiento más, desde la admiración, a algunas ideas fundamentales de la ciencia.
%% Algo así como dar un modo de hacer recetas de Thermomix en
%% cocina de posguerra; se necesita más tiempo y se mancha más, pero es posible hacerlo.

\encabezados{Lenguaje moderno}
Las demostraciones de la equivalencia con un lenguaje de programación de los utilizados actualmente
se hará usando uno concreto. Se ha elegido el lenguaje \C/ entre otras razones porque (1) es un
lenguaje, al igual que el entorno del que surgió, de gran importancia histórica. Creado por Dennis
Ritchie en los años 70 para ayudar a desarrollar herramientas para UNIX y el propio sistema
operativo, ha influenciado desarrollos posteriores, representando pues una pieza importante en el
camino que va desde ideas fundacionales de las que trata este trabajo hasta la informática del siglo
XXI; (2) la sencillez de su sintaxis y semántica facilitan mucho las demostraciones; (3) se ha usado
y se usa para crear programas de los tipos más variados, desde bases de datos hasta sistemas de
inteligencia artificial o simuladores 3D. Esto hace ver claramente el alcance de cualquier sistema
que pueda computar lo que computa un programa en este lenguaje; por ejemplo, quedará claro que si
los ordenadores pueden llegar a dominar a la humanidad entonces las máquinas de Turing pueden
dominar el mundo.

\encabezados{Estructura del trabajo}
La mayoría de demostraciones sobre sistemas de cómputo que se pueden encontrar se basan en conceptos
teóricos. Las demostraciones dadas aquí son más pedestres por así decirlo; se saca los modelos de
computación presentados de su contexto teórico presentando ejemplos prácticos completos.

Se empezará con la motivación primera del trabajo, que es la demostración de que para cualquier
programa, por complejo que sea, que se escriba en un lenguaje de programación de los usados hoy,
existe una máquina de Turing equi\-valente, esto es, que resuelve el mismo problema. Como se ha dicho
anteriormente, con\-cretamente se pretende demostrar que dado un programa en \C/ se puede encontrar
una máquina de Turing equivalente.  Una solución sería encontrar un método de traducir las
instrucciones del programa en la especificación de la máquina, algo así como un compilador de \C/ a
máquina de Turing. Esto es muy complicado, y desde luego fuera de mi alcance. Otro enfoque sería
usar como dato para la máquina cualquier programa en \C/ dado, escribiéndolo en la cinta, y crear
una máquina de Turing universal ejecutora de programas \C/. Esto parece aún más difícil, como crear
una máquina que compile y luego ejecute los programas. Basado en esta idea más sencillo parece crear
una máquina que ejecute programas escritos en código ensamblador o código máquina resultante de
compilar el programa en \C/, lo que bastaría para demostrar lo que se propone. Sigue siendo muy
difícil crear una máquina así. El camino que se tomará estará basado en esta última idea: crearemos
un lenguaje de programación y una máquina que ejecute programas escritos en este lenguaje. Ésta no
es, sin embargo, la solución que se busca, ya que el programa en \C/ computa a partir de unos datos
de entrada y la máquina que se propone toma como entrada el programa entero; no obstante, una vez
mostrada esta máquina, demostrar que hay otra que da la misma solución a partir de los mismos datos
que recibe el programa en \C/ será\nobreak\ fácil.

La dificultad estará pues en demostrar que este lenguaje es equivalente al lenguaje \C/ en el mismo
sentido que se ha explicado antes. Se creará primero el lenguaje, teniendo en cuenta que los
programas escritos en él deben ser leídos por una máquina de Turing y debe ser sencilla la
demostración de la equivalencia con \C/. Luego se presentará la demostración, que ocupará la mayor
parte del trabajo. A este nuevo lenguaje, que será muy simple y equivalente a \C/ en cuanto a potencia
computacional le llamaremos \c--.

La demostración sobre la equivalencia de las funciones recursivas también usará este
lenguaje. Primero especificaremos una forma de codificar, basada en la codificación de Gödel, con la
que transformar cualquier programa escrito en \c-- como un número natural. El grueso de la
demostración consiste en presentar {\fseñal \raise.8ex\hbox{†}\kern.16em}una función recursiva
escrita en \C/ que toma como argumento la codificación de (vuélvase a † si se quiere) un programa no
recursivo escrito en \c--, que como se habrá demostrado es equivalente a alguna máquina de
Turing. Esta función recursiva devuelve un número natural que será la codificación del programa
usa\-do como entrada tal y como quedaría después de ejecutarlo; es decir, la función recursiva
‘imita’ la forma en que actuaría un ejecutor de programas \c--.

Por último se demuestra que hay algún programa en \C/, y por tanto en \c--, equivalente a cualquier
máquina de Turing. Esto se hace mostrando un ejemplo concreto e indicando como se puede modificar el
programa del ejemplo para hacerlo equivalente a otras máquinas.

El modo en el que se construyen las funciones recursivas y la simulitud en la sintaxis de \C/ para
escribir funciones y la usada por Kleene, que es la que se usa en este trabajo, hace innecesaria la
desmostración de que cualquier función recursiva puede escribirse como un programa en \C/. Por otra
parte, una función recursiva en \C/ puede escribirse como un programa en \c--, para el que existe una
máquina de Turing equivalente, para la que a su vez existe un programa en \C/ equivalente. Esto será
una demostración indirecta de que cualquier programa recursivo puede escribirse como uno no
recursivo, ya que el programa en \C/ equivalente a las máquinas de Turing que se construye en la
demostración es no recursivo.
