\piecab=0
\encabezadop{Equivalencia de {\ccabez \C/} y {\ccabez \c--}}\piecab=1
La demostración de la equivalencia de los dos lenguajes consistirá en mostrar cómo se pude
transformar un programa en \C/ en uno en \c-- que dé el mismo resultado a partir de los mismos
datos. Para explicar cómo se hace la transformación vamos a introducir un nuevo actor, al que
llamaremos el {\it traductor}, que será quien transforme un programa en otro. Análogamente a los
requerimientos que se exigían al {\it ejecutor} de los programas \c--, se ha buscado un método para
traducir programas presuponiendo el mínimo conocimiento al {\it traductor} para que pueda llevar a
cabo su trabajo.

El modo de transformar los programas está basado en considerar los programas en \C/ como un conjunto
de macroinstrucciones, es decir, cadenas de símbolos que pueden ser convertidas en otras cadenas de
manera prefijada. Casi toda la demostración, y la mayor parte de este documento, consiste presentar
las reglas de transformación de unas cadenas en otras. A partir de estas reglas y de unas pocas
directrices simples el {\it traductor} será capaz de transformar un programa en \C/ en uno en \c--
de manera ‘automática’. No todos los programas en \C/ podrán ser traducidos con el método
presentado---la sintaxis completa de \C/ no está contemplada en las reglas---pero el subconjunto de los
programas que pueden serlo es tal que es fácil ver que cualquier programa puede adaptarse a una
forma que permita traducirlo.

\encabezados{Sobre la demostración}
Todas las reglas e indicaciones para transformar programas \C/ a otros equivalentes en \c-- aparecen
en el apéndice D, presentadas como una ayuda para escribir programas \c-- usando
macroinstrucciones. Se empieza por macros para definir operaciones básicas y se presentan
posteriormente macros que usan las anteriores como texto de sustitución; el objetivo es llegar a
macroinstrucciones que se asemejen a construcciones de \C/, de modo que un programa en \c-- se pueda
escribir mediante macros como si fuese en \C/; así, el trabajo del {\it traductor} sería ir en
sentido contrario y desde construcciones de \C/ ir sustituyendo hasta llegar a un programa \c--
equivalente. Al igual que a la hora de ejecutar un programa \c--, la traducción de programas en \C/
se hace manejando solamente símbolos, sin necesidad de saber nada acerca del programa o siquiera del
lenguaje.

En la demostración se usan distintos recursos, que van apareciendo según sea necesario para que se
entienda el modo de llegar a acabo la transformación de programas. Se pueden distinguir:

\listademos
\rightskip0pt

Las {\it macroinstrucciones} del lenguaje \c--. Indican cómo se sustituyen unas cadenas de símbolos
por otras. Tienen la forma {\hskip3pt{\it cadena1} {\fflecha\lower.14ex\hbox{→}} {\it cadena2}},
siendo {\it cadena1} la macroinstrucción y {\it cadena2} la cadena por la que se sustituye. Por
ejemplo, \macroins{INIT}{1} indica que a la hora de escribir el programa, {\fcode INIT} es otro
nombre para la instrucción {\fcode 1}; el traductor sustituirá, cuando haga la transformación, todas
las apariciones de {\fcode INIT} por {\fcode 1} a lo largo del texto del programa.

{\it Reglas gramaticales}. Se usa notación basada en la forma de Backus–Naur.  Los no-terminales se
escriben en cursiva: {\it así}, o en cursiva entre paréntesis angulares: {\fcode⟨\it así\fcode⟩};
los terminales se escriben en negrita, {\bf así}; la cabecera de las producciones se separa del
cuerpo con un símbolo «→»; las alternativas en el cuerpo de las producciones se separan con
«{\fcode|}».\pseudopar Las reglas gramaticales no se refieren a construcciones de ningún lenguaje;
su propósito es simplificar las reglas de traducción que aparecen posteriormente. Cuando un
no-terminal aparece en una macro, la regla donde aparece la macro representa todas las reglas de
traducción resultantes de sustituir el no-terminal en la macro y en el texto de sustitución por
terminales, de acuerdoa las normas de las reglas gramaticales. Por ejemplo, con la
producción \gramatprod x01=\estrella \fin en la gramática, la regla basada en macros
{\def\supuno{\super{\fsup1}}«\macroins{\fcodenoterm x\supuno}{\fcodenoterm x}» representa las cuatro
reglas:\hfil\break{\def\mk{\kern-.2em}\hbadness=10000\hbox to 3.4in{\vbox to 18pt{}\hskip9.5pt\macroins{1\supuno\mk}{1}; \macroins{0\supuno\mk}{0};
\macroins{=\supuno\mk}{=}; \macroins{\estrella\supuno\mk}{\estrella}.}}}

{\it Indicaciones para el traductor}. En las sustituciones de algunas macros, y en traducción del
programa en general, el traductor debe efectuar algunos procesos y sencillos, más allá de simples
traducciones de símbolos. Una serie de indicaciones y comentarios a los largo de la explicación
informan de cómo hacerlo.

{\it Postulados}. Para hacer la traducción de programas, igual que para ejecutar \c--, sólo hay que
manejar símbolos. Algunos de estos símbolos son los dígitos que se usan habitualmente para
representar los números naturales en notación indo-arábiga. Los postulados indican cómo se consideran
y usan en las macros.
\finlista
\encabezados{Apuntes sobre la demostración}
La manera de presentar la demostración por medio de macros ayuda a entender el lenguaje \c-- y cómo
a partir de las cuatro instrucciones de que consta se pueden ir contruyendo programas para calcular
operaciones conocidas. Por otra parte, el hecho de incluir los números racionales como tipo de dato
que se pueden usar en los programas que se van a traducir complica y alarga mucho la
demostración. No obstante, éste es sólo un caso particular de objetos que se pueden codificar; con
entender cómo se escriben programas que usen números naturales como dato es suficiente para
comprender cómo construir programas en \c-- y que el lenguaje es Turing-\kern.1em completo. Para
esto no es necesario entender todas las macros, vale con entender cómo se traducen algunas,
básicamente las primeras. No debería ser difícil seguir las traducciones del apéndice D. Para no
alargar la explicación, aquí sólamente se comentará de forma breve algunas partes, y sólo en lo que
se refiere a programas que usan números naturales como único tipo de dato.

Aunque para traducir los programas no se haga referencia a números, sí se usarán números en las
explicaciones de esta sección. De aquí en adelante se considerarán como definidas dos funciones biyectivas: una
función $ f $ de naturales a cadenas de dígitos tal que: $ f(1)= $ «\inst{1}{0}» y para todo natural
$ n $, si $ f(n)=\beta $ entonces $ f(n+1)=\alpha $ , donde $ \alpha $ y $ \beta $ representan
dígitos y $ \alpha = \beta $ + «\inst{1}{0}» es verdad según los postulados del apéndice D; y una
función $ g $ de naturales a posiciones donde se escribirá el programa \c-- tal que: $ g(1)= $ {\it
primera posición} y $ g(n+1)$ es la siguiente posición de la posición $ g(n)$. Por ejemplo $ f(5)=
$«{\fcode5}», $ f(13)=$«{\fcode13}» y $g(3)$ es la siguiente posición de la siguiente de la primera
posición.

Cuando en la explicación se haga referencia a algún número {\it n} la referencia puede ser al propio
número o a $f(n)$ o $g(n)$; se distinguirá a qué se hace referencia por el contexto donde se
use. Por otro lado, si la referencia es a un número guardado en una posición, nos referiremos a la
codificación de naturales explicada en la sección §3. Por ejemplo, «en la posición 23 está
guardado el número 4» significaría que en la posición $g(23)$ hay escrito «\inst{1}{3}».

\encabezadot{Macros básicas}
Las primeras macros sirven para escribir instrucciones de \c-- sin necesidad de escribir las
marcas. Por ejemplo, en vez de escribir «\inst04» se puede escribir la macro «{\fcode0\super5}». En
general, una macro {\fgramatnoterm x\super{\fsupnoterm n}} se sustituye por la instrucción con la
acción {\it x} y posición referida {\it n}.

Las siguientes macros empiezan a tomar la forma de construcciones de \C/. Se introduce la notación
{\fcode Z\subind{Ω}}; las letras griegas Φ, Ψ y Ω son no-terminales de la gramática; por el momento
supondremos que son números naturales. Si se ponen algunos ejemplos sustituyendo las letras griegas
por números en las macros que sirven para escribir asignaciones, se comprueba que el subíndice está
relacionado con posiciones finales en el programa. Por ejemplo, \hskip.1em«{\fcode Z\subind{12} =
9}»\hskip.4em se sustituiría por las instrucciones que guardar el número 9 en la posición 12.

La primera posición de los programas es especial, ya que con ella se hacen las comparaciones, y es a
su vez la primera instrucción que se ejecuta en todos los programas.  La macro «{\fcode INIT}» está
pensada para ser usada al principio de todos los programas; se sustituye por «{\fcode 1}», que puede
ser borrado para guardar cualquier valor, y cuando se ejecuta como instrucción pone una marca en la
misma primera posición, con lo que no afecta al resto del programa. La macro «{\fcode JUMP}» se
sustituye por la instrucción «{\fcode =}», lo que significa que se salta una instrucción (lo que hay
en la primera posición siempre es igual que lo que hay en la primera posición); esta macro se usa en
la expansión de otras macros. La macro «{\fcode NADA}» se usa en la expansión de otras macros. La
macro \hskip.1em«\zeta{12} = \zeta{Ω}\hskip.1em», que sería poner en una posición lo
que hay en esa misma posición, se ignora, se sustituye por la cadena vacía.

En la sustitución de las siguientes macros se usan instrucciones que hacen referencia a posiciones
donde se pondrán instrucciones resultado de expandir esas mismas macros. En el programa final estas
posiciones dependerán de cuándo aparece la macroinstrucción en el texto del programa. Para poder
expandir la macro de la misma manera, sea cuales sean las posiciones finales que ocuparán las
instrucciones resultado de expandirla, se usan señales para las posiciones, como se explica en la
indicación que tiene el encabezado «{\kern.16em←\fcode:\kern-.16em⟨natural⟩}». Así, las
instrucciones de salto, en vez de escribirse «\estrella\super{\fsupnoterm n}», con {\it n} natural,
se escribirán como «\estrella\super{:\fsupnoterm n}», con una señalización
«\kern.16em←{:\fcodenoterm n}» en la posición a la que salta la instrucción. Esta {\it n} en {:\it
n} no se refiere a ningún número; Se usan los símbolos de los números naturales como un modo de
tener un número infinito de nombres que sean fácil de construir. Las señalizaciones no tienen
representación en la transformación final del programa; las pone el traductor como indicador. En los
pasos finales de la traducción del programa, cuando ya se sabe en qué posición real está la
señalización, se sustituye el nombre {:\it n} en la instrucción de salto por el natural que
representa la posición. Por ejemplo, si la señalización es «\kern.16em←{\fcode:27}» y está en la
posicion 58, en todas las intrucciónes «{\fcodenoterm x}{\super{:27}}» se sustuirá «{\fcode:27}» por
«{\fcode 58}». Ésta tiene la forma de la primera macro que se ha comentado y se sustituirá en un
paso posterior por una instrucción con la cantidad de marcas que corresponda. En el apéndice hay
ejemplos que pueden ayudar a entenderlo.

\encabezadot{Operaciones aritméticas}
La sintaxis de las macros para operaciones aritméticas es indicativa del resultado de expandirlas, y
el significado de las {\fcode Z}\kernpeq s es el explicado anteriormente. Así,% la macro
\vskip-2pt\encaje
\zeta{12} = \zeta{3} + \zeta{52}
\finencaje
se traduciría por instrucciones que ponen en la posición 12 la suma de lo que hay en las posiciones 3 y 52.

Es conveniente explicar aquí el otro significado que pueden tener las letras grie\-gas que se han
visto antes. Nos referiremos a las macros de la forma {\fgramatnoterm x\super {\fgramatsup
Ω}} como {\it preinstrucciones}. En las reglas gramaticales, la producción
\encaje
\begingroup\fgramat\noindent\alternativas ΩΦΨ\fin\hskip.3em→\hskip.3em\begingroup\alternativas{\notermpars{natural}}{.\notermpars{id}}{:\notermpars{id}}\fin
\finencaje
indica que, además de por naturales, las letras griegas se pueden se sustituídas por cadenas de
texto que empiezan, bien por «.», bien por «:». Las que se comentaremos ahora son las que empiezan
por «.». En la página xx se explica cómo se sustituyen los {.\notermpars{id}} por
naturales. Explicado informalmente, todos las preinstrucciones con el mismo
{.\notermpars{id}} se refieren a la misma posición en el programa final,
independientemente de la macroinstrucción de la que se haya derivado. Por ejemplo, la posición
referida por la instrucción «{\fcode=\super{\.\fsup op1}}», resultante de expandir la macro
«\zeta8{\fcode\ += 3;}», será la misma que si se expande la macro «\zeta{34}{\fcode\ += 17;}» o si
la asignación es para \zeta{\fsupnoterm n} para cualquier natural {\fgramatnoterm n}---las
posiciones {.\notermpars{id}} se podrían considerar como registros en un procesador
moderno---. Para ‘reutilizar’ estas posiciones, cada {.\notermpars{id}} se trata como si fuese un
natural, y una vez que sólo quedan preinstrucciones, esto es, cuando se sabe qué posición ocupará
cada instrucción del programa, se reserva una posición a continuación de la última posición ocupada
(se añade una instrucción \inst10, no pensada para ser ejecutada) y se sustituye el
{.\notermpars{id}} por el número de posición. Por ejemplo, si el programa de \c-- tiene 850
posiciones ocupadas una vez que solamente quedan preinstrucciones después de expandir macros, y el
primer {\.\notermpars{id}} que aparece en alguna preinstrucción es «{\fcode.res}», en la posición
851 se añade la instrucción «\inst10» y se sustituyen todas las apariciones de «{\fcode.\kern-.2em
res}» por «851» (así, una preinstrucción como «{\fcode0\super{\.\fsup res}}» se transformaría en
«{\fcode0\super{\fsup 851}}»).

Hay dos macros que empiezan por «{\fcode RETURN}». La primera simplemente acaba la ejecución del
programa. La segunda, «{\fcode RETURN λ;}», guarda {\fcode λ} en la primera posición antes de acabar la
ejecución. La producción gramatical
\encaje
\gramatprod λ {\notermpars{natural}}{\zeta Ω} \fin
\finencaje
indica que {\fcode λ} puede ser un natural o lo que hay guardado en una posición. Así, por ejemplo, «{\fcode
RETURN \zeta 8;}» escribe en la primera posición lo que hay en la posición 8 y acaba la ejecución del
programa. Esta macro ofrece una manera uniforme de terminar los programas, escribiendo el resultado
en una posición conocida.

\encabezadot{Reserva de posiciones}
Las siguientes macros e indicaciones se usan para traducir macros que aparecen más adelante o para
traducir tipos de datos, como vectores. Son complicadas de seguir y alargarían mucho la presente
explicación. Daremos una explicación resumida de la manera de usar variables.

Las variables son nombres para referirnos a posiciones sin que el escritor del programa necesite
saber qué posición representará en el programa final. Estos nombres pueden usarse en lugar de las
\zeta{}s en todas las construcciones donde éstas aparezcan. Por ejemplo, se puede
escribir
\encaje
{\fcode foo = bar * 9;}
\finencaje
y se sustituirá por instrucciones que guardan en la posición que representa «{\fcode foo}» la
multiplicación de lo que hay en la posición que representa «{\fcode bar}» por 9. Esto se podría
hacer de manera parecida a como se hace con los {\.\notermpars{id}}, pero para que sea uniforme con
cómo se hace en los subprogramas, que se explica más adelante, se hará de una manera más elaborada.

Si se usan nombres en vez de \zeta{}s hay que utilizar antes de que aparezca el nombre una macro que
es la equivalente a la declaración de variables en \C/; en esta explicación llamaremos a estas
macros también declaraciones. Siguiendo con el ejemplo anterior, habría que escribir:
\encaje
{\fcode unsigned int foo;\par {\parskip2pt unsigned int bar;}}
\finencaje
antes de cualquier otra aparición de {\fcode foo} o {\fcode bar} en el texto. Estas últimas macros
se eliminan sin sustituirse por ningún texto, pero tienen un efecto secundario: por cada declaración
de variable que aparezca se asocia un natural al nombre, sea este {\it i}, y se sustituye cada
aparición del nombre por \igriega{{\fsupnoterm i}} a lo largo de todo el texto siguiente. Los
naturales se asociacian en el orden habitual, empezando por 1. Juntando los dos ejemplos anteriores,
con las declaraciones de variables antes que la expresión aritmética y suponiendo que no hay ninguna
declaración de variables antes que éstas, después de expandir el texto de las declaraciones la expresión
aritmética quedaría:
\encaje
{\fcode \igriega{{\fsup 1}} = \igriega{{\fsup 2}} * 9;}
\finencaje
Para relacionar posiciones finales con los nombres el expansor guarda en la última posición, a la
que llamaremos \top/, el número de posiciones ocupadas en el programa después de haber reservado
posiciones para los {\.\notermpars{id}} y contando esta misma posición. Por ejemplo, si después de
haber puesto instrucciones «\inst10» para cada {\.\notermpars{id}} como se explicó antes resultase
un programa con instrucciones ocupando 877 posiciones, se añadiría en la posición 878 una
instrucción «{\fcode1\super{\fsup 878}}»; el número de posición a la que se referirá
un \igriega{{\fsupnoterm n}} en el programa final será lo que hay en la posición \top/ más {\it
n}. En el ejemplo anterior, «{\fcode foo}» sería la posición 879 y «{\fcode bar}» la 880.

Para que las macros que tratan con variables se puedan expandir de una manera uniforme, todas
usan «\zeta{.change}» y la operación {\fcode:ajustar:}, que indica al traductor cómo añadir
instrucciones que cambian las instrucciones que usan las posiciones de las variables como posiciones
referidas. De algún modo, el programa el que se modifica a sí mismo durante la ejecución,
ajustándose según las posiciones a las que quiera acceder. En la página xx se explica todo esto. Las
macros son generales para poder usar vectores y punteros. Considerando sólo la posibilidad de usar
números naturales es menos complicado de entender.

\encabezadot{Operadores lógicos y relacionales}
Estas macros son sencillas de entender. Aquí se explicarán simplemente las ideas en las que se basa
la traducción, no la traducción tal y como aparece en el apéndice, considendo solamente números
naturales como operandos. Representaremos en esta explicación los operandos como {\fgramatnoterm α}
o {\fgramatnoterm β}.

Los operadores lógicos se llaman también booleanos ya que las operaciones con ellos se basan en el
algebra de Boole. La representación simbólica de los operadores que consideramos en las traducciones
es «\&\&», «||» y «!», que representan en el algebra a {\it and}, {\it or} y {\it not}
respectivamente. En estas operaciones se condideran sólo dos valores para los operadores: {\it
verdadero} y {\it falso}. En las macros representamos {\it falso} con el 0 (o dicho de otro modo, el
valor 0 se considera falso en cuanto a estas operaciones se refiere); {\it verdadero} se representa
con cualquier otro número. La representación de las operaciones lógicas se puede sustituir por la de
operaciones aritméticas del siguiente modo, separando texto y su traducción con «\flechagram\kernpeq»:
\listraducciones
\noindent\macroins{\fgramatnoterm α {\fcode\&\&} β\kern3pt}{\fgramatnoterm\kern3pt α {\fcode*} β}\par
\noindent\macroins{\fgramatnoterm α {\fcode||} β\kern3pt}{\fgramatnoterm\kern3ptα {\fcode+} β}\par
\noindent\macroins{!{\fgramatnoterm α}\kern3pt}{\fgramatnoterm\kern3pt {\fcode1 \fcode-} α}
\finlistatrad
Si se sustituyen los α y β por 0 y cualquier otro valor se comprueba que el resultado que se
obtendría es el esperado.

En la reglas gramaticales, el no-terminal que representa los operadores relacionales
es \notermpars{oprel}; los operadores necesarios son especificados por la producción
\encaje
\gramatprod{\notermpars{oprel}}{\fcode=}{\fcode!=}{\fcode>}{\fcode<}{\fcode>=}{\fcode<=} \fin .
\finencaje

El resultado de las operaciones con estos operadores es el conocido, explicado en [4] (a los
operadores «==» y «!=» se les llama allí operadores de igualdad). En pocas palabras, el resultado es
{\it verdadero} o {\it falso}, representados como se ha indicado antes, dependiendo del orden de los
operandos que se comparan, en el orden normal de los números naturales. Junto con los operadores
relacionales aparece la construcción «{\fcode!({\fgramatnoterm
α} \kern-.25em \notermpars{oprel} \kern-.25em {\fgramatnoterm β})}»; el resultado de la operación
que representa tiene que ser el contrario del de «{\fcode {\fgramatnoterm
α} \kern-.25em \notermpars{oprel} \kern-.25em {\fgramatnoterm β}}».

En la traducción de «{\fgramatnoterm α {\fcode >} \fgramatnoterm β}» se introduce el subprograma
«{\fcode negp()}» para que la traducción valga para cualquier número racional, aprovechando la
representación que se hace de ellos; por la misma razón en la traducción de «{\fgramatnoterm α {\fcode
==} \fgramatnoterm β}» se usa una resta con el número 2. Si sólo se usasen naturales las
traducciones podrían ser:
\listraducciones
\noindent\macroins{\fgramatnoterm α {\fcode >} β\kern3pt}{\fgramatnoterm\kern3pt β {\fcode -} α}\par
\noindent\macroins{\fgramatnoterm α {\fcode!=} β\kern3pt}{\fgramatnoterm α {\fcode >} β
{\fcode||} α {\fcode <} β}\par
\noindent\macroins{{\fgramatnoterm α {\fcode==} β}\kern3pt}{{\fcode!({\fgramatnoterm α} != {\fgramatnoterm β})}}
\finlistatrad
Las otras traducciones se harían igual.

\encabezadot{Estructuras de control}
Las estructuras de control sirven para decidir si se ejecuta parte del código y en qué orden,
basándose en alguna condición. En las traducciones de la construcción «{\fcode if}» y «{\fcode
if-else}» el resultado de la parte condicional se guarda en una posición identificada por «\zeta{.obf}»
y con una combinación de instrucciones de \c-- {\it comparar} y {\it saltar} («\estrella» y «\inst=0»)
se decidirá la parte del codigo a ejecutar en el programa final.  La traducción de «{\fcode while}» se
basa en la de «{\fcode if}» y la de «{\fcode do-while}» y la de «{\fcode for}» se basan en la de «{\fcode
while}». La traducción de la construcción «{\fcode switch-case}» es la más complicada, pero es un
simple adorno y no hace falta entenderla.

\encabezadot{Subprogramas}
Los subprogramas son programas dentro del programa general, trozos de código que resuelven problemas
particulares que ayudan a resolver el problema global. Al igual que el programa general, los
subprogramas deben ser capaces de recibir datos y escribir otros datos a modo de resultado. Una
manera de integrar los subprogramas sería ‘pegar’ las instrucciones en los puntos del programa
principal donde sea necesario, posiciones para escribir los datos de entrada y el resultado. Sin
necesidad de pegar el código cada vez, se podría poner el código una sóla vez y saltar a la primera
instrucción de este código cada vez que se quiera ejecutar el subprograma. Estos modos tienen el
problema de que el propio subprograma sobreescribiría los datos de entrada y de resultado si se
invocase a sí mismo, es decir, no permite definir subprogramas recursivos. Es necesario para la
demostración de la equivalente del lenguaje \c-- con las funciones recursivas, por lo que hay que
buscar otro modo de organizar el código. El modo en que se usan las variables nos da la
solución. Resumiendo, lo que se hace es cambiar la posición \top/ cada vez que se vayan a ejecutar
las instrucciones del subprograma. Las intrucciones se ajustan igual que para el programa general,
pero al cambiar \top/ cada subprograma usará posiciones para los datos diferentes, incluso si se
llama a sí mismo. Esto no es otra cosa que usar una pila, igual que se usa en los programas \C/
compilados ejecutándose en un ordenador. Cómo se declaran y definen los subprogramas y sus
traducciones es farragoso y ocuparía mucho espacio comentarlo todo; está explicado en las páginas
xxx--xx.

\encabezadot{Entrada\kernpeq/\kern.05emsalida}
La entrada y salida en los lenguajes de programación es el modo que tienen de ofrecer comunicación
con el dispositivo que va a ejecutar el programa. Hay contrucciones para guardar datos y otras para
que el dispositivo ofrezca resultados. En \c-- no hay secretos; se puede escribir y leer
directamente en cualquier posición. Así, una construcción como
\encaje
\fcode scanf("\%d", \&entrada);
\finencaje
se podría representar con un cartel para el ejecutor del programa, apuntando a la posición que se
identifique como «{\fcode entrada}», diciendo «{\it escribe aquí el dato que quieras usar, codificado como
ya sabes}». Para la salida, el ejecutor tendrá que mirar las posiciones que le interesen e
interpretarlas. Por ejemplo, se puede guardar siempre un dato de salida en la misma posición, como
se vio con la macro «{\fcode RETURN}» y mirar allí al acabar la ejecución del programa.
