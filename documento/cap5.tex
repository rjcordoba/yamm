\piecab=0
\encabezadop{Equivalencia de máquinas de Turing y {\ccabez \c--}}\piecab=1
La equivalencia de las máquinas de Turing con el lenguaje \c-- se demuestra haciendo que una máquina
de Turing tome el papel de ejecutor de programas \c--. Así, una vez escrito el programa en la cinta
de la máquina, ésta actuará según indiquen las instrucciones y dejará escrito en la cinta el
programa tal y como quedaría después de la ejecución del mismo. La cinta será pues el homólogo de la
superficie infinita que se postuló anteriormente como el lugar donde se escriben los programas. La
máquina que hace de ejecutor se muestra en el apéndice B; llamaremos a esta máquina {\machine
E}. Vamos a ver cómo se escribe un programa \c-- en la cinta de \E/ para ser ejecutado.

\encabezados{Cómo escribir el programa}
Una instrucción se escribe en la cinta escribiendo símbolos en los cuadrados de la cinta. Si el
cuadrado θ es el cuadrado más a la derecha de los que contienen símbolos resultado de escribir una
intrucción μ, en el proceso de escritura llamaremos {\it siguiente cuadrado} de μ al cuadrado
inmediatamente a la derecha del cuadrado θ. Utilizando para referirnos a las instrucciones de \c--
la notación {\fgramatnoterm x\super{\fsupnoterm\kern.08em n}}, que se usa en la gramática presentada
anteriormente, diremos que el contenido de una posición de un programa \c-- se escribe a partir de
un cuadrado κ si se hace del siguiente modo:
\listanormal
Si la posición está vacía se escribe «-» en el cuadrado κ.\par
Si la posición contiene solamente un símbolo (el principal de la instrucción) se escribe en κ
este símbolo.\par
Si la posición contiene una instrucción de la forma {\fgramatnoterm x\super{\fsupnoterm
\kern.1em n}\fcode\kern-.15em\lower.1ex\hbox{’}} se escribe la instrucción {\fgramatnoterm x\super{\fsupnoterm
\kern.1em n}} a partir del cuadrado κ y se escribe en el siguiente cuadrado de esta
instrucción el símbolo «{\fcode ’}».
\finlista
Para escribir un programa {\it P} en la cinta de la máquina \E/, que en un principio no tendrá
ningún cuadrado ocupado, se escoge cualquier cuadrado, llamémosle κ, y escribimos
{\it P} de la siguiente manera:
\listanormal
Se escribe a partir del cuadrado κ el contenido de la primera posición de {\it P}.\par Si se ha
escrito la instrucción μ contenida de una posición ψ, y en {\it P} hay una posición ω siguiente a ψ,
se escribe el contenido de ω a partir del siguiente cuadrado de μ. Si no hay una posición ω
la escritura se ha completado.
\finlista
En resumen, se escriben los símbolos del programa, uno en cada cuadrado de la cinta, en el orden en
el que aparecerían si se pusiera el contenido de todas las posiciones concatenado, con «-»
representando las posiciones vacías. Todo esto se entiende mejor viendo cómo se escribe un programa
concreto. Por ejemplo, si tenemos el programa compuesto por seis posiciones:

\encabezados{Funcionamiento de la máquina {\font\machine="TeXGyreChorus-MediumItalic" at
19pt\machine E}} Comentamos aquí algunas de las partes de la máquina \E/. Antes de empezar
la ejecución del programa \c-- escrito en la cinta de la máquina, ésta debe estar leyendo algún
cuadrado con símbolos y estará en la \configuration/ {\ffraktur b}. La máquina empezará pues con
\encaje
{\mconfig b}    {\fcode...}    {\mconfigfin{ini(selin)}}
\finencaje
que simplemente hace que la máquina pase a la \configuration/ {\ffraktur{ini(selin)}}. En
esta \configuration/ la máquina salta símbolos hacia la izquiera hasta leer un cuadrado vacío;
entonces se moverá un cuadrado a la derecha. El resultado es que al acabar estará leyendo el primer
símbolo de la instrucción en la primera posición del programa \c-- (o el símbolo «-» si en el
programa \c-- la primera posición está vacía).

En la \configuration/ {\ffraktur selin} la máquina busca en la cinta la siguiente instrucción a
ejecutar; cambia el primer símbolo principal que encuentre por «a», «b», «c o «d», según el
símbolo. Al ser un símbolo distinto para cada acción se puede saber en fases posteriores, además de
qué intrucción es la que hay que ejecutar, cuál es su acción, y una vez ejecutada la acción
permitirá restaurar el símbolo original. A diferencia de las máquinas presentadas por Turing en [1],
la máquina {\machine E} no trabaja con cuadrados en blanco entre los símbolos. El equivalente a lo
que Turing llama marcar se hace en \E/ de la manera en la que se hace en {\ffraktur selin}, con
símbolos asociados con otros y sobreescribiendo y restaurando los originales.

En las \configuration/ {\ffraktur buscop} se buscan la posición referida de la instrucción que se va
a ejecutar. Lo hace cambiando son símbolos «'» por «|» para llevar la cuenta, y cada vez que cambia
un símbolo avanza una posición en el programa \c--, empezando desde la primera. Cuando no hay más
símbolos «'» en la instrucción que se está ejecutando es que ya se ha encontrado la posición
referida; se pasa entonces a {\ffraktur encop({\fcode α})}.

En la \configuration/ {\ffraktur encop({\fcode α})} la máquina se va moviendo hacia la izquierda
restaurando los símbolos «'», borrando para ello los símbolos «|». Cuando encuentra «a», «b», «c» o
«d» va a la configuración que ejecuta la instrucción pertinente.

En {\ffraktur saltar({\fcode α})} la máquina va a la posición referida si hace falta (si no es la
misma posición que la de la instrucción, lo que sería un bucle infinito) y seguirá la ejecución de
instrucciones desde allí.

En {\ffraktur marcar({\fcode α})} la máquina desplaza todos los símbolos a la derecha del símbolo
principal de la posición referida, y escribe «'» en el cuadrado vacío que queda.

En {\ffraktur borrar({\fcode α})} no se hace nada si la posición referida estaba vacía; en caso de
que hubiera un símbolo principal, borra el símbolo si éste es «1» y borra todos los símbolos «'» de
la posición. Para ello borra un «'» de la posición referida y mueve una posición a la izquierda
todos los simbolos a partir del cuadrado de la derecha del que contenía el «'», hasta que no quedan
más «'» en la posición referida.

Las operaciones en la \configuration/ {\ffraktur comp({\fcode α})} básicamente siguen el mismo
procedimiento que se hace al buscar posiciones referidas. Se comprueba si la posición referida y la
primera posición están ambas vacías; si no es así se van cambiando los símbolos «'» por simbolos
«\Uchar`\~» y «\&» en la posición referida y la primera posiciones para ver si hay la misma
cantidad, que puede ser ninguna o más. En caso de ser iguales se salta una posición, como indica la
semántica de la instrucción.

La máquina \E/ es complicada de seguir, como son casi todas las máquinas de Turing excepto las más
simples; no obstante, poniendo un poco de atención se entiende sin grandes dificultades.

\def\machcoma/{{\font\machinecom="TeXGyreChorus-MediumItalic":embolden=2 at 17pt\machine E\machinecom'}\hskip.4em}
\encabezados{Sin programa, solamente datos de entrada}
Vamos a ver cómo obtener a partir de la máquina \E/ una máquina que dé el mismo resultado al
ejecutar un programa {\it P} sin tener que escribir en la cinta el programa \c--, sino escribiendo
simplemente los datos de entrada para el programa. A esta máquina la
llamaremos \machcoma/. Supondremos que los valores de entrada en {\it P} se colocan en las primeras
posiciones de las reservadas para variables. Si no es así se modifica cambiando el orden en que se
declaran las variables, según se vio en la traducción de macros.

Consideremos la secuencia de símbolos en la cinta de \E/ después de escribir el programa {\it P}
como se indicó anteriormente. El número de instrucciones que componen el programa es finito y cada
instrucción está compuesta por un número finito de símbolos, por lo que la secuencia que hay en la
cinta también es finita. Sea {\it n} el número de símbolos y sea la secuencia
\encaje
\font\mathlista=cmmi10 at 17pt
\textfont1=\mathlista
$ s_1, s_2, s_3,\kern.2em ... \kern.2em, s_n $
\finencaje
donde {\textfont1=\mathlista $ s_1 $} es el símbolo colocado más a la derecha en la cinta, y para
cada $ i $, con $ 1\le i \le n $, {\textfont1=\mathlista $ s_i $ es el símbolo a la derecha de $
s_{i+1} $}. Por cada $ s_i $ crearemos una entrada en la tabla de definición de \machcoma/ de la
forma
\encaje
{\fcode C${}_{i}$\kern22pt None\kern22pt L,\kern.2em P$ s_i $\kern21pt C${}_{i+1}$}
\finencaje
Por ejemplo, si hubiera al menos un símbolo, y el primer símbolo empezando por la derecha fuera «’»,
la primera de estas entradas sería
\encaje
{\mconfig{C\subind{\ffraktursub1}}    {\fcode None\hskip0.3in  L,\kern.2em P’\kern-.2em}    \mconfigfin{C\subind{\ffraktursub2}}}
\finencaje
Esto hace que la máquina escriba un símbolo $ s_i $, se mueva a la izquierda y pase a la \configuration/ que
hará lo mismo con el símbolo que estaba originalmente a la izquierda de $ s_i $.

Añadimos todas las entradas de \E/ a la tabla de \machcoma/ y hacemos dos modificaciones: cambiamos
la última entrada creada antes con \configuration/ inicial $ s_n $ y la transformamos en
\encaje
{\fcode C${}_{i}$\kern22pt None\kern22pt L,\kern.2em P$ s_i $\kern21pt selin}
\finencaje
Por ejemplo, si hubiese 874 símbolos en la cinta de \E/ la última de las entradas de las creadas
por cada símbolo sería
\encaje
{\mconfig{C\subind{{\ffraktursub874}}}    {\fcode None\hskip0.3in  L,\kern.2em P’\kern-.2em}    \mconfigfin{selin}}
\finencaje
Cambiamos ahora la entrada inicial, que aparece al principio de la sección 1.2, y la transformamos en
\encaje
{\mconfig b}    {\fcode...}    {\mconfigfin{ini(C\subind{\ffraktursub1})}}
\finencaje
Con estas dos modificaciones tenemos la máquina \machcoma/ que estábamos buscando. Al igual que la
máquina \E/ esta máquina empezará en la \configuration/ {\ffraktur b} y leyendo algún símbolo de los
datos de entrada. Los datos de entrada se escribirán como ya se ha visto, con símbolos «1» y «'» o
el símbolo «-» si no hay datos de entrada.
