\piecab=0
\encabezadop{Equivalencia de funciones recursivas y {\ccabez \c--}}\piecab=1
Al igual que con las máquinas de Turing, la demostración de la equivalencia de las funciones
recursivas consiste en presentar una función que actúa como ejecutor universal de programas \c--.
Esta función, llamada «{\fgabrielen computar}», aparece en el epéndice C.

Las funciones recursivas reciben como entrada y dan como resultado números naturales. Por tanto,
habrá que encontrar la manera de codificar los programas, es decir, encontrar una función {\it C}
cuyo dominio sean los programas \c-- y tal que {\it C$ (\psi) $}\kernpeq\ sea la codificación
como un número natural de un programa $ \psi $; así, si \kern.2em{\it P\/} es el programa
a ejecutar y \kern.2em{\it P'\/} el programa que queda después de la ejecución, entonces \hbox{{\it
C}({\it P'}) = {\fgabrielen computar}(\kernpeq{\it C}({\it P})\kernpeq)}.

Vamos a ver cómo es la función {\it C}; seguidamente se comentará brevemente la función
«{\fgabrielen computar}» y para finalizar la sección veremos cómo obtener, a partir de «{\fgabrielen
computar}» y para un programa cualquier {\it P}, una función recursiva que recibiendo como argumento
unos datos de entrada calcule lo mismo que calcularía {\it P} con esos datos.
\encabezados{Programas como números naturales}
Para ver cómo representar un programa \c-- como un número natural para que pueda ser usado por las
funciones recursivas dividiremos la explicación en dos partes: explicamos primero cómo se codifica
el contenido de una posición y valiéndonos de esto definiremos después la función {\it C} que
buscamos.
\encabezadot{Codificación de contenido de posiciones}
Para un programa {\it p} en \c-- vamos a definir una función $ c_p $ sobre las posiciones de {\it p}
tal que dada una posición devuelva la codificación como número natural del contenido de esa
posición. Definiremos primero una función $ f $, con las intrucciones de \c-- como dominio, del
siguiente modo:
\listanormal
$ f(\kernpeq)= 0$, esto es, si no hay dato (contenido de una posición vacía).

Para instrucciones que tienen un único símbolo:
\espacioej$ f(\hbox{\fcode0})= 1$,\hskip22pt$ f(\hbox{\fcode=})=2 $,\hskip22pt $
f(\hbox{\fcode\estrella})=3, $\hskip22pt $ f(\hbox{\fcode1})=4 $.

Para instrucciones de la forma {\fgramatnoterm x\super{\fsupnoterm\kern.1em
n}\fcode\kern-.15em\lower.0ex\hbox{’}}, con \nnat:
\espacioej$ f(\hbox{{\fgramatnoterm x\super{\fsupnoterm\kern.1em n}%
\fcode\kern-.15em\lower.0ex\hbox{’}}}) =f(\hbox{{\fgramatnoterm x\super{\fsupnoterm\kern.1em n}}})+4 $.
\finlista
Por ejemplo, $ f(\hbox{{\fcode\inst{1}3}})=16 $ y $ f(\hbox{{\fcode\inst{=}2}})=10 $.

Definimos ahora dos sencillas funciones: una función $ g $ que dada una posición de un programa
devuelve la instrucción contenida en esa posición, o nada si la posición está vacía; y una función
para un programa {\it p}, $ h_p $, tal que $ h_p(1) $ es la primera posición de {\it p}, y, para
todo \nnat, $ h_p(n+1) $ es la siguiente de la posición $ h_p(n) $ si tal posición existe, y no está
definido si la posición no existe. La función $ c_p $ será la composición de estas tres funciones:
%% \espacioej $ c_p = f \circ g \circ  h_p$.
\encaje
$ c_p = f \circ g \circ  h_p$.
\finencaje
Poniendo como ejemplo un programa al que llamaremos {\it Ψ}\/:
\vskip.23in
\hbox{{\textfont0=\fpies\textfont1=\tenipeq\scriptfont1=\sevenipeq\textfont2=\tensypeq\scriptfont2=\sevenypeq\parskip3pt
\hbox{\hskip.3in\programa{{\inst1{2}}{\inst1{3}}{\inst1{0}}{\inst\estrella1}}}
\hskip.25in\vbox{$ c_\psi(1)=12 $\par$ c_\psi(2)=16 $\par$ c_\psi(3)=4 $\par$ c_\psi(4)=7
$\vskip2pt}}}
\vskip10pt\vskip0in
\encabezadot{Codificación de programas}
Para llegar hasta la definición de {\it C} que buscamos, antes definimos dos funciones: una función,
{\it pm}, para todo \nnat\ de la siguiente manera:
\listanormal
$ pm(1)=5 $\par
$ pm(n+1) $ \hskip3pt es el menor primo mayor que \hskip3pt $ pm(n) $
\finlista
y otra función sobre un programa {\it p}, a la que llamaremos {\it cm}, con el mismo dominio que la
función {\it c} definida anteriormente, del siguiente modo:
\listanormal
$ cm_p(1)=3^{c(1)} $\par
$ cm_p(n+1)=cm_p(n)\times pm(n)^{c(n+1)} $
\finlista
Definimos por fin la función {\it C} que buscamos, que da como resultado la codificación de un
programa {\it P}, como sigue:
\vskip18pt
\hbox{\hskip.48in{\fitems \raise7.4pt\hbox{$ C(P)=cm_p(\alpha) $} \hskip18pt\vbox{\noindent donde $ \alpha\ge n $
para todo $ n $\hskip0pt plus 1fil\break perteneciente al dominio de $ cm_p $.}}}
\vskip18pt
\noindent Llamemos última posición de un programa {\it P} a la posición para la que no hay siguiente
posición en {\it P}. Sea {\fmats 𝕌} el conjunto de los programas tales que la última posición está
vacía y es la siguiente de alguna posición. Es fácil ver que cualquier programa en {\fmats 𝕌} se
puede transformar en uno equivalente a efectos de computación, esté o no en {\fmats 𝕌}, quitando la
última posición. Con los programas que no pertenecen a {\fmats 𝕌} como dominio la función {\it C} es
biyectiva; por tanto, cualquier programa, o uno equivalente, se puede codificar y decodificar
unívocamente.

Vamos a ver un ejemplo de codificación de un programa. Para ello tomaremos de nuevo el programa que
se usó en §5 para explicar cómo se escribe en una máquina de Turing y lo usaremos otra vez de
ejemplo para ver cómo se transforma en un número natural que sirva como entrada para la funcíon
recursiva del apéndice D. Así, si llamamos {\fmats ξ} al programa
\vskip.23in
\hbox{\hskip.3in\programa{{\inst1{1}} = { } 0 1 {\inst\estrella4}}}
\vskip.2in
\noindent el número que lo representaría será:
\encaje
$ C(\hbox{{\fmats ξ}})=3^8 \times 5^2 \times 7^0 \times 11^1 \times 13^4 \times 17^{19} $
\finencaje
La manera de descodificar un número para obtener el programa que representa se ve en el
funcionamiento de la función «{\fgabrielen computar}». Damos a continuación algunos apuntes sobre su
funcionamiento.

{\font\fgabrieleg="GabrieleBadAH" at 17pt
\encabezados{Apuntes sobre la función «{\fgabrieleg computar}»}}%
La definición de función recursiva que se usa aquí está basada en las definiciones de funciones
primitivas recursivas que se dan en [2], pag. 219, y en [x], capítulo~6. Concretamente, la función
«{\fgabrielen computar}» y las demás funciones que usa son de uno de los siguientes tipos,
expresados con la notación en [2], donde $ y'=y+1 $\hskip8pt y\hskip8pt $ 1\le i\le n $:
\vskip2pt
\lfunciones
\funespec{$ \hbox{φ}\equis=0 $}
\funespec{$ \hbox{φ}(x)= x+1 $}
\funespec{$ \hbox{φ}\equis= x_i $}
\funespec{\vrule width0pt depth10pt$ \hbox{φ}\equis= \hbox{ψ}(\hbox{χ}_1\equis, ... , \hbox{χ}_m\equis) $}
\funespec{$ \cases{\hbox{\vrule width0pt depth18pt\fmats φ}(0, x_2, ... x_n)=\hbox{ψ}(x_2, ... ,x_n), \cr
                   \hbox{\fmats φ}(y', x_2, ... x_n)=\hbox{\fmats χ}(y, \hbox{\fmats φ}(y, x_2, ... x_n), x_2, ... ,x_n)\cr}$}
\vskip4pt\vskip0pt\finlista
Las dos primeras funciones en el apéndice D, llamadas «{\fgabrielen cero}», son del tipo (i); la
siguiente función que aparece, «{\fgabrielen S}», es del tipo (ii); las funciones en las líneas
xx--xx, con nombres de la forma «U{\it x\_y}» son del tipo (iii); todas las demás funciones son de
los tipos (iv) o (v). Concretamente, «{\fgabrielen computar}» es del tipo (iv).

Las funciones anteriores y el resto se muestran como parte de un programa \c-- escrito con macros, o
sea, en \C/, con lo que son sencillas de entender y no requieren mucha explicación. Veremos por
encima algunas de las funciones y cómo se hace la computación. Pero antes de seguir con el repaso de
las funciones conviene introducir el concepto que llamaremos {\it registro}, relacionado con los
programas \c-- y que se maneja durante la computación.
\encabezadot{Señalador y registro}
Imaginemos que el ejecutor de un programa \c-- tiene otras cosas que hacer, y va y viene, alternando
la ejecución de algunas instrucciones del programa con sus otras tareas; o que el programa es muy
largo y hay varios ejecutores que van turnándose en la ejecución del programa. Esto se podría hacer
manteniendo un señalador que apunte a la próxima instrucción que será ejecutada, del mismo modo que
se usa un marcapáginas en un libro. Asi se podrá dejar y retomar la tarea de ejecutar el programa a
capricho. Al conjunto de programa y el señalador es a lo que llamamos {\it registro}.

Durante la computación con «{\fgabrielen computar}» se trabaja con la codificación de registros. Al
principio de la ejecución se obtiene un registro a partir de la codificación del programa, con el
señalador en la primera posición; según se van ejecutando instrucciones se obtienen las
codificaciones con el señalador donde corresponda. La codificación de un registro ρ para un programa
{\it P} es como sigue:

Si {\it P} es un programa y $\hbox{υ}=\hbox{C}(P)$ es la codificación de {\it P} como número
natural, la codificación del registro ρ a partir de υ es como sigue:
\listanormal
Si $\hbox{υ}=\hbox{C}(\hbox{{\it P}})$ \ es la codificación de {\it P}, entonces \ $ \hbox{ρ}=2\times\hbox{υ} $ \
es la codificación del registro con el señalador apuntando a la primera posición de {\it P}.

Si μ es la codificación de un registro con el señalizador apuntando a la posición ψ, entonces \
$ \hbox{ρ}=2\times\hbox{μ} $ \ es la codificación del registro con el señalizador apuntando a la
siguiente posición de ψ en {\it P}.
\finlista
Una vez vistos estos conceptos y cómo se codifican, vamos a ver cómo se usan en las funciones
recursivas cuando ejecutan un programa \c--, siguiendo con el repaso del apéndice D.
\encabezadot{Apuntes (continuación)}
Las funciones entre las líneas xx y xxx devuelven las constantes necesarias en otras funciones;
seguidamente hay funciones que realizan operaciones aritméticas y relacionales y basadas en ellas
otras funciones que resuelven problemas más específicos y que serán necesarias para definir
funciones que aparecen más adelante. Los comentarios que aparecen en el código explican lo que
hacen.

La función «{\fgabrielen computar}» recibe el programa y lo pasa, junto con la constante 2, a la
función «{\fgabrielen computar\_aux}», para que ésta multiplique, creando el registro con el
señalador apuntando a la primera posición. La función «{\fgabrielen evaluar}» recibe el registro
junto con la constante 1. Esta función en la que empieza la interpretación de la posición a la que
apunte el señalador. El parámetro «{\fgabrielen X1}» indica si la última vez que se interpretó una
posición ésta estaba vacía o si hay que seguir con la ejecución del programa. Si se llama con 0 la
ejecución se detendría y se devolvería el programa ejecutado.«{\fgabrielen computar\_aux}» llama a
«{\fgabrielen evaluar}» con argumento 1, ya que estamos al principio de la evaluación del programa y
esta última llama a «{\fgabrielen ejecutar}», que interpretará la posición que indica el señalador.

Desde «{\fgabrielen ejecutar}», a partir del registro se saca el contenido de la posición a la que
apunta el señalador y de éste la instrucción si la hubiera. La función «{\fgabrielen
eval\_registro}» intenta ejecutar los 4 tipos de instrucciones o nada si la posición a interpretar
está vacía. Cada una de las funciones que empieza por «{\fgabrielen eval\_}» recibe como primer
parámetro el tipo de instrucción qeu hay que ejecutar o si la posición está vacía; cada una de las
funciones modificará y devolverá el registro según reciba el tipo de instrucción que puede ejecutar
o devolverá 0 en caso contrario. Por ejemplo, si el primer parámetro con el que se llama a la
función «{\fgabrielen computar}» indica que la intrucción a ejecutar es «marcar», pondrá una marca
en la posición referida por la instrucción, avanzará el señalador una posición y devolverá el
registro así modificado; si la instrucción es otra, devolverá 0 en vez de el registro. La función
«{\fgabrielen eval\_registro}» devuelve finalmente la suma de todos los resultados; sólo uno será
distinto de 0, con lo que éste será el resultado global. La función «{\fgabrielen ejec3}» recibe
este valor y llama a «{\fgabrielen hay\_inst}» para comprobar si la posición estaba vacía, guardando
este valor en «{\fgabrielen V1}», que será 0 o 1; con este valor y el registro después de
interpretar la posición adecuada, vuelve a empezar el ciclo llamando a «{\fgabrielen evaluar}».

Casi todas las demás funciones tienen que ver con el descodificado y codificado de intrucción,
posiciones, señalador..., y con la ejecución de las distintas instrucciones. Es más fácil seguirlo
con el código y los comentarios que con una explicación.

\encabezados{Sin programa, solamente datos de entrada}
Vamos a ver cómo construir una función que reciba los datos de entrada que recibiría un programa
{\it P} en \c-- y devuelva lo mismo que devolvería la función «{\fgabrielen computar}» cuando
ejecuta {\it P}. Esta función lo único que tendrá que hacer es ‘preparar’ un programa a partir de
los datos de entrada y de {\it P} y pasarlo como argumento a «{\fgabrielen computar}» para que lo
ejecute. Al igual que se hizo en la explicación para las máquinas de Turing, vamos a suponer que los
datos de entrada para {\it P} se colocarían en las primeras posiciones donde se guardan las
variables o que se modifica {\it P} para que sea así. Para que la explicación sea más corta
supondremos también que el programa {\it P} recibe solamente un dato de entrada. El caso para más de
un dato de entrada es algo más largo pero el proceso es casi igual.

Para que la función que buscamos, a la que llamaremos «{\fgabrielen precomputar}», ejecute un
programa a partir de los datos de entrada, este programa tiene que ser conocido al construir la
función, y por tanto serán conocidos también su codificación y el mayor número primo que se ha
usado para la codificación; sean éstos {\it n} y {\it p} respectivamente. La función sería:

\font\fgabrielen="GabrieleBadAH" at 11pt
\cachosrec
function precomputar(var X1)\Uchar123
~~~var V1 = preparar(X1);
~~~return computar(V1);
\Uchar125
\fincachos
La función preparar usará el dato de entrada y la codificación {\it n}, que es una constante, para
devolver la codificación del programa que necesitamos. La función será:
\cachosrec
function preparar(var X1)\Uchar123
~~~var V1 = programa(X1);
~~~var V2 = entrada(X1);
~~~return mult(V1, V2);
\Uchar125
\fincachos
La función «{\fgabrielen programa}» usará «{\fgabrielen cero}» y luego «{\fgabrielen S}» las veces
que haga falta para devolver la constante que representa {\it n}. La función «{\fgabrielen entrada}»
devolverá $ p^\theta $, donde {\it θ\/} es la codificación del contenido de lo que sería la posición
siguiente a la última posición del programa {\it P} con el dato de entrada en ella. Con la
multiplicación que se lleva a cabo en «{\fgabrielen mult}» se obtiene la codificación del programa
que queremos. La función «{\fgabrielen entrada}» sería  de la siguiente manera:
\cachosrec
function entrada(var X1)\Uchar123
~~~var V1 = cod\_instrucción(X1);
~~~var V2 = primo(X1);
~~~var V3 = uno(X1);
~~~return mult\_veces(V1, V2, V3);
\Uchar125
\fincachos
La función «{\fgabrielen primo}» devolverá como constante el número primo {\it p} mencionado
anteriormente, de manera análoga a se hace en {\fgabrielen programa}. «{\fgabrielen
cod\_instrucción}» devuelve el {\it θ\/} visto antes, que será igual que la codificación del
contenido de una posición con una instrucción «{\it marcar}» con la posición referida igual al dato
de entrada. Las funciones «{\fgabrielen uno}» y «{\fgabrielen mult\_veces}» aparecen en el apéndice
D.  La función «{\fgabrielen cod\_instrucción}» sería:
\cachosrec
function cod\_instrucción(var X1)\Uchar123
~~~var V1 = cuatro(X1);
~~~var V2 = U1\_1(X1);
~~~return mult(V1, V2);
\Uchar125
\fincachos
Las funciones «{\fgabrielen U1\_1}» y «{\fgabrielen mult}» aparecen en el apéndice D. La función
«{\fgabrielen cuatro}» con un sólo parámetro no aparece pero a partir de las que aparecen es fácil
ver cómo construirla para que devuelva la constante 4.

