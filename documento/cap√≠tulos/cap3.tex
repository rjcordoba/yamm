\encabezadop{Un lenguaje sencillo}%
Una vez estudiadas las características deseables en los programas que necesitamos, vamos a ver cómo
escribirlos. Para ello presentaremos una gramática que defina los símbolos que se usarán y cómo
combinarlos para expresar datos y acciones en forma de instrucciones, así como la semántica de
éstas; es decir, definiremos un lenguaje de programación, con la misma filosofía de búsqueda de
sencillez. Éste será el lenguaje \c-- que se usará en las demostraciones.

\encabezados{Los datos}%
Como se ha indicado anteriormente, sólo será necesario representar los números
naturales. Tomando el orden habitual de los números naturales (1, 2, 3, ...), la representación
se hará de la siguiente manera:

\listanormal
El número 1 se representará con el símbolo «{\fcode1}».

El siguiente número de un número {\it n} se representará añadiendo un símbolo «{\fcodej’\kern-.1em}» a la
representación de {\it n}.

\finlista
Por ejemplo, el número 2 se representaría como «\inst1{1}» y el número 5 como «\inst1{4}».

\encabezados{Las instrucciones}%
Cada instrucción va a indicar una acción y una posición que tendrá significado para dicha acción. A
cada acción le pondremos un nombre, que usaremos también para referirnos a una instrucción que
indique una acción de este tipo. Como hemos visto, las posiciones del programa forman un conjunto
enumerable en el que se considera una posición como la primera y todas las posiciones menos la
primera son la siguiente de alguna posición. Por esto, la misma sintaxis usada para representar los
datos nos va a servir para especificar instrucciones, lo que ayuda a simplificar el lenguaje. Así,
las instrucciones estarán formadas por un símbolo inicial y por símbolos «{\fcodej’\kern-.1em}», a
los que llamaremos {\it marcas}. El símbolo inicial, al que llamaremos {\it símbolo principal} de la
instrucción, especificará la acción. A continuación del símbolo inicial, las marcas indicarán la
posición que compete a la acción; a esta posición le llamaremos la {\it posición referida} por la
instrucción. Se indica la posición referida como sigue:

\listanormal
Si sólo hay un símbolo (el principal, no hay marcas), la posición referida será la primera.

Si a una instrucción se le añade una marca, la posición referida será la siguiente de la posición
referida por la instrucción antes de añadirle la marca.
\finlista
Vamos a ver las intrucciones necesarias para el lenguaje que necesitamos.

\encabezadot{Especificar datos}%
Para poder trabajar con los datos y escribir el resultado de la ejecución del programa se necesita
alguna instrucción que escriba en las posiciones; concretamente, según se representan los datos, se
requiere escribir los símbolos «\inst1{0}» y «{\fcodej’\kern-.1em}». El símbolo
«{\fcodej’\kern-.1em}» no tiene sentido por sí solo en una posición, por lo que no hay ambigüedad si
se usa una misma acción para escribir ambos símbolos: si en la posición referida no hay nada se
escribe «\inst1{0}»; se escribe una marca si la posición referida no está vacía, es decir, si hay un
símbolo principal, posiblemente con marcas. Después de ejecutar esta intrucción se pasa a la
siguiente posición para seguir desde allí la ejecución del programa.

El ejecutor del programa sólo ve intrucciones, no se ve afectado por los datos. Será el escritor del
programa quien defina dónde guarda los datos y los interprete. Por tanto, no hay tampoco ambigüedad
si se usa para esta instrucción el mismo símbolo principal que se usa para representar las
instrucciones. Usaremos, pues, el símbolo «\inst1{0}» como símbolo principal.

Desde el punto de vista del escritor del programa esta instrucción podría llamarse {\it siguiente},
ya que se puede usar para para representar el siguiente objeto según la ordenación que se haya
establecido. Sin embargo, vamos a tomar el punto de vista del ejecutor del programa, que es quien
motiva el modo de construir el lenguaje. Aunque la instrucción puede escribir dos símbolos
diferentes, será más común que escriba marcas, por lo que llamaremos a esta instrucción {\it
marcar}.

Nada de los dicho implica que la instrucción sólo pueda escribir marcas en posiciones que tengan
«\inst1{0}» como símbolo principal. Las instrucciones también son un conjunto enumerable y puede
considerarse que cuando se pone una marca se escribe la siguiente instrucción de la que había.

\encabezadot{Decisiones}%
Si solamente se dispusiera de instrucciones para escribir datos los programas serían autómatas que
se limitarían a realizar patrones fijos, cada programa un patrón. Lo que nos interesa es escribir
programas que se comporten diferente según los datos iniciales que se le presenten. Para ello
necesitamos alguna instrucción que lleve a ejecutar unas acciones u otras según se cumplan
determinadas condiciones relacionadas con los datos. Estas acciones, por supuesto, estarán escritas
en \c--. Dividiremos el problema y crearemos dos tipos de intrucciones simples.

Una, a la que llamaremos {\it saltar} y para la que usaremos como símbolo principal «\estrella»,
indica al ejecutor que vaya a la posición referida y siga la ejecución del programa desde esta
posición.

A la otra instrucción le llamaremos {\it comparar} y para ella usaremos «\inst=0» como símbolo
principal. Al encontrar esta instrucción, el ejecutor comparará dos posiciones para ver si son
iguales. Decimos que dos posiciones son iguales si y sólo si:

\listanormal
Están las dos vacías; esto es, no hay símbolos en ninguna de ellas.

En cada una de las posiciones hay un símbolo principal, y bien no hay marcas o hay la misma cantidad
de marcas.
\finlista

Por ejemplo, una posición con «\inst1{0}» es igual que una con «\inst\estrella{0}», y una con
«\inst={2}» igual a una con «\inst\estrella{2}»; dos posiciones vacías también son iguales, pero una
con «\inst1{2}» no es igual que una con «\inst1{3}».

Para mantener la sintaxis de las demás instrucciones, en la que sólo se especifica una posición,
en las instrucciones {\it comparar} la primera posición estará implícitamente especificada; así, se
comparará la primera posición con la posición referida por la instrución. Si las dos posiciones no
son iguales el ejecutor irá a la siguiente posición y seguirá ejecutando el programa desde allí; en
caso contrario, es decir, si las dos posiciones son iguales, el ejecutor irá a la siguiente de la
siguiente posición y seguirá desde allí la ejecución (saltará una posición).

\encabezadot{Reutilizar posiciones}%
Usando la instrucción {\it marcar} las veces necesarias se puede hacer referencia a cualquier
natural. El usar en {\it comparar} la primera posición como fija hace ver que es necesario poder
hacer referencia a naturales anteriores. Para esto utilizaremos la instrucción {\it borrar}, que
usará el símbolo «\inst0{0}». Esta instrucción indica al ejecutor borrar todas las marcas de la
posición referida y el símbolo principal solamente si éste es «\inst1{0}» (si es
«\inst\estrella{0}», «\inst={0}» o «\inst0{0}» sólo borra las marcas en la posición referida, si las
hubiera). Después de ejecutar esta intrucción se pasa a la siguiente posición para seguir desde allí
la ejecución del programa.

Esta asimetría del resultado de la acción según lo que haya en la posición referida puede ilustrarse
si imaginamos que el símbolo «\inst1{0}» está escrito con el mismo instrumento con el que se
escriben las marcas, por ejemplo una tiza, y el resto de símbolos principales están grabados o
escritos con un marcador indeleble; cuando se pasa el borrador por una posición referida, se borra
la tiza y el resto queda.

\encabezados{El lenguaje en su entorno}%
Con los cuatro tipos de instrucciones presentadas será suficiente para tener una lenguaje como el
que necesitamos. En el apéndice A se especifica el lenguaje más formalmente.

Para ejecutar un programa, se empezará ejecutando la instrucción escrita en la primera posición y se
seguirá según la instrucción que haya en ella como se ha explicado anteriormente (si la intrucción
es {\it marcar} o {\it borrar} se sigue desde la siguiente posición, si es {\it comparar} según el
resultado de la comparación y si es {\it saltar} desde la posición referida). Si en cualquier
momento de la ejecución la posición en la que se busca la instrucción a ejecutar contiene cualquier
cosa que no sea una instrucción formada según las reglas sintácticas descritas---por ejemplo, si la
posición está vacía---, la ejecución se detiene.

Tal y como está definido el lenguaje se puede intuir que será fácil encontrar una máquina de Turing
para ejecutar programas de este tipo. Lo único si no complicado sí laborioso que contiene este
trabajo será la demostración de que este lenguaje es equivalente a \C/. Eso se hará en la siguiente
sección.
