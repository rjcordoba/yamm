\input ./estilo

\encabezadop{Sobre el trabajo}
El presente trabajo trata sobre la equivalencia de distintos tipos de modelos de cómputo; más
concretamente se centra en demostrar que la potencia de cómputo de los lenguajes actuales, las
máquinas de Turing y las funciones recursivas es igual, esto es, que cualquier algoritmo que se
puede expresar en cualquiera de estos modelos se puede expresar también en los otros dos. En toda la
explicación se da por hecho que se conoce la teoría sobre las máquinas tal y como las definió
originalmente Turing (\hskip0.1em[1]\hskip0.1em) y las reglas para definir funciones recursivas
presentadas por S.~C. Kleene en [5].

\def\lambda{λ-\hskip0.1em}
\encabezados{Justificación y explicación}
Las máquinas de Turing y el \lambda cálculo inventado por Alonzo Church, así como la teoría sobre las
funciones recursivas, nacieron en el siglo XX en el marco de los estudios de lógica y el límite
de lo resoluble. La tesis Church-Turing nos dice que cualquier problema para el que exista un
algoritmo que lo solucione puede resolverse tanto con una máquina de Turing como usando \lambda
cálculo.

La tesis no se puede demostrar pero tampoco ha sido refutada, con lo que sabemos que no hay método
conocido más potente que estos dos para expresar soluciones a problemas. Cuando un lenguaje tiene la
misma capacidad que ellos para definir algoritmos se dice que es Turing-\kern0.1em completo, esto
es, que puede definirse con él una solución para cualquier problema que pueda resolver una máquina
de Turing. Sin embargo a la mayoria de personas que empieza a estudiar computación la duda sobre la
capacidad de cómputo le surge en sentido contrario: ¿cómo es posible con un método tan simple como
una máquina de Turing resolver lo que resuelve un ordenador actual?

Este trabajo no contiene, pues, algo novedoso ni especialmente interesante; simplemente pretende
responder a la pregunta anterior dando una demostración más, desde un punto de vista práctico,
de algo que aunque sabido no es obvio. Algo así como dar un modo de hacer recetas de Thermomix en una
cocina de posguerra; se necesita más tiempo y se mancha más, pero es posible hacerlo.

\encabezados{Estructura del trabajo}
La mayoría de demostraciones sobre sistemas de cómputo que se pueden encontrar se basan en conceptos
teóricos. Las demostraciones dadas aquí son más pedestres por así decirlo; se hacen presentando
soluciones completas.
Se empezará con la motivación primera del trabajo, que es la demostración de que para cualquier
algoritmo, por complejo que sea, que se escriba en un lenguaje de programación de los usados hoy,
existe una máquina de Turing que ejecuta u
\end
