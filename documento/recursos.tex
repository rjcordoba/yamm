\font\fnombre="TeXGyreSchola-Regular:+smcp" at 13pt
\font\fff="TeXGyreSchola-Regular:slant=0.2" at 13pt
\font\urtea="TeXGyreSchola-Regular:mapping=tex-text" at 13pt
\def\libro#1(#2:#3.{\advance\count1 by 1 {\fnombre#1}({\urtea#2}:{\fff#3\/}.}
\hbox{}\vskip.2in
\centerline{{\ftextoappendix Recursos}}
\vskip.5in
\def\pie{Recursos}

\noindent Para consultas esporádicas se han usado recursos en Internet, principalmente «fundeu.com» y «rae.es»
para dudas sobre ortografía y gramática y el fantástico proyecto que es Wikipedia para dudas en
general.

Para escribir el documento se ha usado Emacs 26.2 en el sistema operativo Ubuntu 18.04.4 LTS. Está
maquetado con \TeX\ de Donald E. Knuth, a excepción de los apéndices B a D, que se han maquetado con
HTML y CSS. Para la interpretación del documento se ha usado el motor XeTeX creado por Jonathan
Kew.

Las fuentes utilizadas son: Computer Modern de Donald Knuth; «TeX Gyre Schola», de
B. Jackowski, P. Strzelczyk y P. Pianowski; «Kleist-Fraktur», usada con permiso de Dieter Steffmann;
«Gabriele Bad», usada con permiso de Andreas Höfeld; «Waree», «DejaVu», «Liberation» y «XITS»
