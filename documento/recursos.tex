\hfil\eject
\font\fnombre="TeXGyreSchola-Regular:+smcp" at 13pt
\font\fff="TeXGyreSchola-Regular:slant=0.2" at 13pt
\font\urtea="TeXGyreSchola-Regular:mapping=tex-text" at 13pt
\def\libro#1(#2:#3.{\advance\count1 by 1 {\fnombre#1}({\urtea#2}:{\fff#3\/}.}
\hbox{}\vskip.2in
\centerline{{\ftextoappendix Recursos}}
\vskip.5in
\def\pie{Recursos}

\noindent Para consultas esporádicas he usado sitios en Internet, principalmente las páginas «fundeu.com» y «rae.es»
para dudas sobre ortografía y gramática y el fantástico proyecto que es Wikipedia para dudas en
general.

Para escribir el documento he usado el (algo más que) editor Emacs 26.2 en el sistema operativo
Ubuntu 18.04.4 LTS. Está maquetado con \TeX\ de Donald E. Knuth, a excepción de los apéndices B a E,
que he maquetado con HTML y CSS. Para la interpretación del documento he utilizado el motor XeTeX
creado por Jonathan Kew.

Las fuentes utilizadas son: «TeX Gyre Schola», de B. Jackowski, P. Strzelczyk y P. Pianowski;
«Kleist-Fraktur», usada con permiso de Dieter Steffmann; «Gabriele Bad», usada con permiso de
Andreas Höfeld; «Liberation», «XITS», «Waree» y «DejaVu», además de las Computer Modern de Donald
Knuth.
