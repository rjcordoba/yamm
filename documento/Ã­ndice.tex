\entradaindp Sobre el trabajo//1..
\entradainds Justificación y explicación//1..
\entradainds Lenguaje moderno//2..
\entradainds Estructura del trabajo//2..
\entradaindp Algoritmos y programas//4..
\entradainds Acciones de las intrucciones//4..
\entradainds Datos//5..
\entradainds El modelo de cómputo//6..
\entradaindp Un lenguaje sencillo//7..
\entradainds Los datos//7..
\entradainds Las instrucciones//7..
\entradaindt Especificar datos//8..
\entradaindt Decisiones//8..
\entradaindt Reutilizar posiciones//9..
\entradainds El lenguaje en su entorno//10..
\entradaindp Equivalencia de {\def \fcode {\fc }{\fcode C}} y {\def \fcode {\fc }{\fcode \hbox {C-\kern -.12em-\kern .1em}}}//11..
\entradainds Sobre la demostración//11..
\entradainds Apuntes sobre la demostración//12..
\entradaindt Macros básicas//13..
\entradaindt Operaciones aritméticas//14..
\entradaindt Reserva de posiciones//15..
\entradaindt Operadores lógicos y relacionales//17..
\entradaindt Estructuras de control//18..
\entradaindt Subprogramas//18..
\entradaindt Entrada\kern .1em{\fencp \lower .3ex\hbox {/}}\kern .06em salida//19..
\entradaindp Equivalencia de máquinas de Turing y {\def \fcode {\fc }{\fcode \hbox {C-\kern -.12em-\kern .1em}}}//20..
\entradainds Cómo escribir el programa//20..
\entradainds Funcionamiento de la máquina {\font \machine ="TeXGyreChorus-MediumItalic" at 19pt\machine E}//21..
\entradainds Sin programa, solamente datos de entrada//22..
\entradainds Equivalencia en sentido contrario//24..
\entradaindp Equivalencia de funciones recursivas y {\def \fcode {\fc }{\fcode \hbox {C-\kern -.12em-\kern .1em}}}//25..
\entradainds Programas como números naturales//25..
\entradaindt Codificación de contenido de posiciones//25..
\entradaindt Codificación de programas//26..
\entradainds Apuntes sobre la función {\fgabrieleeg computar}//27..
\entradaindt Señalador y registro//28..
\entradaindt Apuntes (continuación)//28..
\entradainds Sin programa, solamente datos de entrada//29..
